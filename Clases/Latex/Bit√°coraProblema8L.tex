\documentclass[letterpaper, 12pt, oneside]{article}%para dar formato al documento
\usepackage{amsmath}
\usepackage{graphicx}
\usepackage{xcolor}
\usepackage{enumitem}
\usepackage[utf8]{inputenc}

\title{\Huge Problema 8L}
\author{Diego Armando Santillán Arriaga}
\date{22/01/19}

\begin{document}
\maketitle
\newpage
\section*{¿Cómo resolví el problema?}
Escogí que el programa devuelva una lista de los números  pares y una de los números impares que hay entre 2 valores  porque considero que es la aplicación más útil de las listas al problema 8.
El programa es muy similar al del problema 8. Con un ciclo while se generan los números que están entre los valores que introdujo el usuario. Cada uno se clasifica como par o impar con condicionales if y else y se agrega a la lista acorde. La función imprime ambas listas y texto que indica a que corresponden, pares o impares. 
\end{document}


