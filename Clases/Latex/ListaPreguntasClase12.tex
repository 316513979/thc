\documentclass[letter paper, 12pt, oneside]{article}
\usepackage{amsmath}
\usepackage{graphicx}
\usepackage{xcolor}
\usepackage{enumitem}
\usepackage[utf8]{inputenc}


\title{\Huge Preguntas de la clase 12}
\author{Diego Armando Santillán Arriaga}
\date{26/01/2019}


\begin{document}
	\maketitle	
	\newpage
\textbf{Explica como se genera una lista por compresión en python}
\\
R: Entre corchetes primero se escriben las características que tendrá cada elemento de la lista y después un ciclo for que irá generando los elementos.
\\\\
\textbf{¿Para qué se utiliza la función pprint?}
\\
R: Imprime las listas ordenadas verticalmente.	
\\\\
\textbf{¿Qué comando se utiliza para regresar los elementos de una lista anteriores a otro elemento?}
\\
R: ["nombre de la lista":["índice del elemento a partir del cual queremos que nos regrese los elementos anteriores"]]
\\\\
\textbf{En python ¿Qué es una lista? (Clase, método, atributo, objeto, función, etc)}
\\
R: Una lista es un objeto. 
\\\\
\textbf{Si 2 listas son iguales (tienen los mismos elementos en el mismo orden) pero tienen diferentes nombre ¿se consideran el mismo objeto?}
\\R: No, son distintos objetos.
\\\\
Considere la siguiente lista L=[9000, 7564, 324, -0.001234]. 
\textbf{¿Cuál es el resultado de L.index(324)?}
\\
R: 2.
\\\\
\textbf{¿Para qué sirve la clase de documentos beamer en LaTex?}
\\
R: Para hacer presentaciones de diapostivas.
\\\\
\textbf{¿Para qué sirven los documentos tipo "verbatim" en LaTex?}
\\
R: Se utilizan para escribir sin que LaTex pueda hacer modificaciones o reconocer comandos. 
\\\\
\textbf{¿Para qué se utiliza el comando begin{frame} en LaTex?}
\\
R: Para crear diapositivas.
\\\\
\textbf{¿Con qué comando se escribe el titulo de una diapositiva?}
\\
R: Con frametitle. 

\end{document}