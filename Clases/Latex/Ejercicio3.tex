\documentclass{book}
\usepackage[spanish]{babel}
\usepackage[utf8]{inputenc}
\usepackage{biblatex}
\usepackage{hyperref}

\title{Taller de Herramientas Computacionales}
\author{Diego Armando Santillán Arriaga}
\date{17/Enero/2019}

\begin{document}
\maketitle
\tableofcontents
\section*{Introducción} Este libro nos ayudará a fortalecer nuestros conocimientos sobre la materia de Taller de Herramientas Computacionales. 
\url{www.google.com}
\hyperref[Google]{www.google.com}
\chapter{Uso Básico de Linux}
\section{Distribuciones de Linux}
\section{Comandos}

%Con verbatim se pueden escribir comandos y se mantiene el orden y formato. 
\begin{verbatim}

#!/usr/bin/python2.7
# -*- coding: utf-8 -*-

print "Hoy es miércoles"
print '''Diego Armando Santillán Arriaga, 316513979,
Taller de Herramientas Computacionales,
este es un programa que dice "hoy es miércoles"
'''
x=10.5; y=1.0/3; z=15.3
#x, y, z=10.5, 1.0/3, 15.3
H = """
El punto en R3 es:
(x, y, z)=(%.2f,%g,%G)
""" % (x, y, z)
print H

G="""
El punto en R3 es:
(x, y, z)=({laX:.2f},{laY:g},{laZ:G}
""".format(laX=x, laY=y, laZ=z)

print G

import math as m
from math import sqrt
from math import sqrt as s
from math import *
x=16
x=input("Cuál es el valor al que le quieres calcular la raiz:")
print "la raiz cuadrada de %.2f es %f" % (x, m.sqrt(x))

print sqrt(16.5)
print s(16.5)

\end{verbatim}
% Otra manera de hacer esto es poner \input"nombre del archivo", el archivo tiene que estar en la misma carpeta. Ejemplo: \inputFuncion08.py

%Aquí inician los capítulos del libro 
\chapter{Introducción a LaTex}
\chapter{Introducción a Linux}
\chapter{Introducción a Python}
\section{Orientación a objetos}

\begin{thebibliography}{9}
	\bibitem{Libro}
	\textit{Cualquier cosa}
	Autor blah blah, 2019
\end{thebibliography}
\end{document}




