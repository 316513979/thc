\documentclass[letterpaper, 12pt, oneside]{article}%para dar formato al documento
\usepackage{amsmath}
\usepackage{graphicx}
\usepackage{xcolor}
\usepackage{enumitem}
\usepackage[utf8]{inputenc}

\title{\Huge Clase 09}
\author{Diego Armando Santillán Arriaga}
\date{17/01/19}


\begin{document}
	\maketitle
	\newpage
Para saber cual es el tipo de una variable se escribe type("variable")
type("variable")
El comando tmp (temporal) 
\section{Libros el LaTex}
Entre los formatos para escribir en LaTex está el de libro. El formato de libro nos permite incluir portada, introducción, índice y organización por capítulos, entre otras cosas. 
 
\begin{verbatim}
if a<b:
      tmp = b
       b = a
       a = tmp
r=a%b       
\end{verbatim}
\end{document}