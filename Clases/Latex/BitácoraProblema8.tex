\documentclass[letterpaper, 12pt, oneside]{article}%para dar formato al documento
\usepackage{amsmath}
\usepackage{graphicx}
\usepackage{xcolor}
\usepackage{enumitem}
\usepackage[utf8]{inputenc}

\title{\Huge Problema 8}
\author{Diego Armando Santillán Arriaga}
\date{22/01/19}

\begin{document}
\maketitle
\newpage
\section*{¿Cómo resolví el problema?}
Quería hacer un programa que resultara útil o al menos relevante. Por eso decidí escoger la tarea de hacer una función que contara los números pares e impares que hay entre 2 números. Además bastan algunas pocas modificaciones para que el programa cuente múltiplos de algún otro número diferente de 2.
En el programa se utiliza un ciclo while y condicionales if y else. Con el ciclo while se hace que una variable adquiera los valores que están entre los 2 números que introduce el usuario. Cada uno de estos valores se divide entre 2 con la división entera y el resultado se multiplica por 2. Luego este nuevo resultado se compara con el valor original utilizando los condicionales: si son iguales se suma una unidad al contador de los números pares y si no sucede los mismo pero al contador de los impares. Cabe mencionar que el proceso de la división y multiplicación pudo haberse realizado con la operación "módulo". Al final se devuelven los valores de ambos contadores.
\end{document}