\documentclass[letterpaper, 12pt, oneside]{article}%para dar formato al documento
\usepackage{amsmath}
\usepackage{graphicx}
\usepackage{xcolor}
\usepackage{enumitem}
\usepackage[utf8]{inputenc}

\title{\Huge Problema 9L}
\author{Diego Armando Santillán Arriaga}
\date{22/01/19}

\begin{document}
\maketitle
\newpage
\section*{¿Cómo resolví el problema?}
 Escogí devolver los divisores de un número dado en una lista porque esto permite usarlos después, a diferencia del problema 9. 
 El programa consiste en lo siguiente. Con un ciclo while una variable adquiere los valores desde el 1 hasta el número y se calcula la operación módulo entre ambos. Luego con el condicional if se escogen aquellos valores de la variable para los que la operación módulo tuvo un resido igual a 0, y se agregan a la lista. Al final se devuelve la lista. 
\end{document}


