\documentclass[letterpaper, 12pt, oneside]{article}%para dar formato al documento
\usepackage{amsmath}
\usepackage{graphicx}
\usepackage{xcolor}
\usepackage{enumitem}
\usepackage[utf8]{inputenc}

\title{\Huge Problema 5}
\author{Diego Armando Santillán Arriaga}
\date{22/01/19}

\begin{document}
\maketitle
\newpage
\section*{Definir}
El problema consiste en generar una función que calcule la suma de los números naturales hasta el natural n que introduce el usuario. Esta suma es 0+1+2+...+n. 
\section*{Analizar y delimitar}
Las entradas de la función son números naturales y las salidas también por lo que este es su dominio y rango. Para solucionar el problema utilizaremos una variable que funcione como contador, una variable que vaya acumulando el resultado de las sumas y un ciclo while.  
\section*{Solución}
Inicialmente se tienen 2 variables i=0 y s=0; i es el contador y s acumula el resultado de las sumas. Con el ciclo while i va adquiriendo los valores desde el 0 hasta el n y cada uno se va sumando a la variable s. La última repetición sucede cuando i=n-1 por lo que, en el bloque del ciclo, i se vuelve n y agrega a la suma de los valores anteriores. 

\end{document}