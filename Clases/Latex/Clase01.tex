\documentclass[letter paper, 12pt, oneside]{article}
\usepackage{amsmath}
\usepackage{graphicx}
\usepackage{xcolor}
\usepackage{enumitem}
\usepackage[utf8]{inputenc}


\title{\Huge Clase 01}
\author{Diego Armando Santillán Arriaga}
\date{07/01/2019}


\begin{document}
	\maketitle
\newpage
\section{ Sistemas Operativos}

Un sistema operativo es un conjunto de programas que se encargan de controlar el hardware de una computadora. Gracias al sistema operativo nos es posible utilizar aplicaciones y programas en nuestra computadora. 
Existen múltiples sistemas operativos. Los más populares son \textbf{Windows, iOS y Linux}.Este último se puede obtener en alguna de sus distribuciones(conjunto de software ya configurado): Fedora, Ubuntu o Debian entre otras. Además Linux destaca por ser un sistema operativo de software libre lo cual implica que es posible modificarlo.

\section{Linux}

Como cualquier sistema operativo Linux tiene un entorno grafico, esto es un "menú" en el que se puede navegar para ejecutar programas, abrir y guardar documentos y, en general, administrar la computadora. Sin embargo para poder tener un mayor acceso al sistema operativo es necesario utilizar la \textbf{línea de comandos}. Su funcionamiento consiste en escribir y ejecutar códigos especiales que cumplen funciones determinadas.En el caso de Linux a estos códigos se les llama comandos bash mientras que la línea de comandos es un intérprete de comandos bash. Esto se debe a que Linux utiliza el programa informático bash para recibir e interpretar ordenes. 

 \section{Archivos en Linux}
 
 
 En Linux el uso de un archivo esta limitado por ciertos permisos. Hay 3 tipos de permisos, de lectura(r), de escritura(w) y de ejecución(x).
 Estos se otorgan al usuario de la computadora, a grupos con los que se comparten archivos y al público en general. Es decir los permisos rwx se asignan a cada uno de estos "sujetos". Es importante tener en cuenta que cada permiso tiene un número asociado: el de lectura 4, el de escritura 2 y el de ejecución 1. Al final se suman los números de todos los permisos que tiene un archivo para un sujeto y después con los dígitos resultantes se forma un número de 3 cifras que indica los permisos que tiene un archivo.  
 
 
 Para conocer los permisos de un archivo y su número de 3 cifras correspondiente se utiliza el comando \textbf{ls-l  "nombre del archivo"}. Por otro lado para cambiar los permisos de un documento se utiliza \textbf{chmod nnn "nombre del archivo"}. En el lugar en donde están las enes se debe de colocar el nuevo número de 3 cifras, según los permisos que se quieran asignar a cada "sujeto".
 
 \newpage
 \section{Variables de entorno}
 
 Las variables de entorno definen la forma en que trabaja el sistema y permiten modificar como "se comporta" el shell. Un ejemplo es la variable PATH que permite buscar variables binarias. El comando \textbf{set} se utiliza para observar las variables de entorno. 
 
\section{Comandos útiles}



\textbf{uname -a}: arroja información sobre el kernel (programa fundamental del sistema operativo encargado de gestionar los recursos del hardware).

\textbf{touch "nombre de archivo"}: crea un archivo vacío.

\end{document}



