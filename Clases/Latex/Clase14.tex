\documentclass[letter paper, 12pt, oneside]{article}
\usepackage{amsmath}
\usepackage{graphicx}
\usepackage{xcolor}
\usepackage{enumitem}
\usepackage[utf8]{inputenc}


\title{\Huge Clase 13}
\author{Diego Armando Santillán Arriaga}
\date{24/01/2019}


\begin{document}
	\maketitle	
	\newpage
\section*{Notas}
\subsection*{Tuplas:}
las tuplas son listas que no se pueden modificar. Son iguales que las listas solamente que se reemplazan los corchetes por paréntesis. 

\subsection*{elif:} Funciona como un else solo que se coloca para que se admitan más casos. En lugar de solo dar dos casos posibles, el del if y el de else, se dan más casos, uno por cada elif. El último caso debe de ir en el bloque de un else. 
\\\\
\textbf{Ejemplo:}
\begin{verbatim}
>>>if x<0:
       instrucción
   elif x=0:
       instrucción
   elif 0<x<1:
       instrucción
   else:       # la última opción es un else
       instrucción
\end{verbatim}

\subsection*{NumPy:}
Para instalarlo en fedora se ocupa \textbf{pip install numpy}. Es una biblioteca que provee hartas funciones y herramientas científicas.   
     	
\subsection*{list('cadena'):}
Este comando transforma cada carater de la cadena en una cadena que inserta en una lista. 
\textbf{Ejemplo:}
list('ATGC') = ['A','T', 'G', 'C']

\subsection*{Porcentaje s:}
Se utiliza para colocar una variable del tipo cadena dentro de una cadena de texto.
\end{document}