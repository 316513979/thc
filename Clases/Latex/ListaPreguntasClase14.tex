\documentclass[letter paper, 12pt, oneside]{article}
\usepackage{amsmath}
\usepackage{graphicx}
\usepackage{xcolor}
\usepackage{enumitem}
\usepackage[utf8]{inputenc}


\title{\Huge Lista de preguntas clase 14}
\author{Diego Armando Santillán Arriaga}
\date{24/01/2019}


\begin{document}
	\maketitle	
	\newpage
\textbf{¿Qué es una tupla?}
\\
R: Una lista que no se puede modificar. 
\\\\
\textbf{Menciona 3 operaciones que se puedan hacer a listas y tuplas:}
\\
R: len, zip e index.
\\\\
\textbf{¿Para qué sirve elif?}
\\
R: elif se utiliza para introducir agregar más casos a un bloque if y else. 
\\\\
\textbf{¿Qué es NumPy?}
\\
R: Es una biblioteca que provee múltiples funciones y herramientas científicas.
\\\\
\textbf{Escriba el resultado del comando: L=list('ATGC'):}
\\
R: L=['A', 'T', 'G', 'C']
\\\\
\textbf{¿Cómo se coloca una variable tipo string en una cadena?}
\begin{verbatim}
R: Se escribe %s en el lugar de la variable y luego % ("variable") 
al finalizar la cadena.
\end{verbatim}

\end{document}

