\documentclass[letter paper, 12pt, oneside]{article}
\usepackage{amsmath}
\usepackage{graphicx}
\usepackage{xcolor}
\usepackage{enumitem}
\usepackage[utf8]{inputenc}

\title{\Huge Clase 03}
\author{Diego Armando Santillán Arriaga}
\date{09/01/2019}


\begin{document}
	\maketitle
\newpage
\title{\huge\textbf{Como subir archivos a Github}}

Para subir un archivo a github debe de estar guardado en el repositorio que creamos en nuestra computadora. Una vez hecho esto desde el repositorio ejecutamos los siguientes comandos en el orden indicado:

 \textbf{1. git init}: esto conecta el repositorio de nuestro ordenador con Github.

\textbf{2. git add *}: este comando se usa para agregar archivos nuevos y modificaciones. El asterisco indica que se agregaran todos los archivos y modificaciones que se hayan hecho. Tambien es posible agregar un solo archivo utilizando \textbf{git add "nombre del archivo"}
 

\textbf{3. git commit}: con este comando se ordena a git que registre modificaciones y archivos nuevos por lo cual pide agregar un comentario en el que indiquemos que cambios y archivos se agregaron al repositorio. Podemos escribir directamente el comentario utilizando \textbf{git commit -m "comentario"}. 

\textbf{4. git push}: la ejecucion de este comando sube todos los archivos nuevos y modificacionespero para ello, cuando se nos indique debemos escribir nuestro nombre de usuario y contrasena de Github.


\title{\huge\textbf{Comandos importantes}}


\textbf{mkdir}: crea un directorio


\textbf{mkdir -p directorio1/directorio2/...}: crea directorios anidados.


\textbf{cat "nombre del archivo"}: muestra el contenido de un archivo.


\textbf{ls}: muestra directorios.


\textbf{ls -la}: Muestra informacion sobre directorios y archivos (en el extremo izquierdo "d" significa directorio y "-" archivo).


\textbf{file "nombre de archivo"}: indica el tipo de un archivo. 

\textbf{file *}: indica el tipo de archivo de todos los archivos de un directorio. 


\title{\huge\textbf{Proceso para resolver un problema}}


\textbf{1. Definir o entender el problema}


\textbf{2. Analizarlo y delimitarlo}: analizar como se puede resolver y que instrumentos se tienen disponibles para ello.


\textbf{3. Revisar soluciones ya existentes}


\textbf{4. Describir la solucion detalladamente}


\textbf{5. Presentar una solucion general}


\end{document}