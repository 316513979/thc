\documentclass[letterpaper, 12pt, oneside]{article}%para dar formato al documento
\usepackage{amsmath}
\usepackage{graphicx}
\usepackage{xcolor}
\usepackage{enumitem}
\usepackage[utf8]{inputenc}

\title{\Huge Preguntas de la clase 13}
\author{Diego Armando Santillán Arriaga}
\date{26/01/19}

\begin{document}
	\maketitle
	\newpage
\textbf{¿Cómo se llaman las 2 partes que componen una función recursiva?}
\\
R: Base de recursividad y regla de recursividad. 
\\\\
\textbf{¿A qué se refiere el ámbito local de validez de una variable?}
\\
R: Las variables locales son las que pertenecen a una función y están dentro de su bloque. 
\\\\
\textbf{¿Qué es una variable global?}
\\
R: Las variables globales son las que se declaran fuera de una función y no tienen sangría. 
\\\\
Considere la siguiente función:
\begin{verbatim}
def suma(n):
    if n==1:
        return 1
    else:
        return n+suma(n-1)   
\end{verbatim}
\textbf{¿Es recursiva?}
\\
R: Sí.
\\\\
\textbf{¿Cuál es la diferencia entre una función recursiva y una función iterativa?}
\\
R: Las funciones iterativas utilizan ciclos for o while para repetir una intrucción mientras que las recursivas hacen que un resultado se base en otro anterior para repetir una instrucción.
\\\\
\textbf{Escriba la función mostrada anteriormente de forma iterativa o recursiva según corresponda:}
\\
R:
\begin{verbatim}
def suma(n):
    i=1
    s=0
    while i<=n:
        s=s+i
        i=i+1
    return s   
\end{verbatim}

\end{document}
