\documentclass[letter paper, 12pt, oneside]{article}
\usepackage{amsmath}
\usepackage{graphicx}
\usepackage{xcolor}
\usepackage{enumitem}
\usepackage[utf8]{inputenc}


\title{\Huge Clase 04}
\author{Diego Armando Santillán Arriaga}
\date{10/01/2019}


\begin{document}
	\maketitle
\newpage
\title{\huge\textbf{Planteamiento de problemas}}


En el siguiente problema aplicaremos el proceso de resolucion de problemas visto en la clase pasada: se lanza una pelota al aire y se debe estimar su posicion a traves del tiempo. 
La funcion que describe la posicion de la pelota es \textbf{y(t) = v0*t-$\frac{1}{2}*t$} donde y es la posicion en un tiempo de terminado, v0 es la velocidad inicial de la pelota y t es el tiempo. De aqui se deduce que la trayectoria de la pelota sera una parabola y que el dominio de la funcion son todos los reales. Sin embargo vamos a delimitar el dominio al intervalo de tiempo \textbf{[0s , 6.93s]} pues no consideraremos las medidas negativas de posicion. Por otro lado consideraremos v0 = 34 m/s y g = 9.81 m/$s^{2}$. De aqui procedemos a calcular los valores en los que estamos interesados en python. 


\title{\huge\textbf{Calculos en python}}

\textbf{Idle}: Idle es un entorno de desarrollo integrado que se utiliza para programar en python. Idle abre \textbf{python shell}. A su vez un  \textbf{shell} es un interprete de comandos especializado. De esta manera se debe de abrir Idle para poder utilizar python. Para ello se utiliza el comando \textbf{idle}.

\textbf{Observaciones importantes para hacer calculos en python:}

~Para que una \textbf{division} arroje un resultado decimal es necesario que alguno de los operandos este escrito como decimal. De lo contrario redondea el resultado al menor entero. \textbf{Ejemplo:} $\frac{1}{2} = 0$ pero $\frac{1.0}{2} = 0.5$ o $\frac{1}{2.0} = 0.5$.

~Para guardar un programa es necesario copiarlo del shell a un documento de texto plano utilizando la pestana de File del menu de editor. Al guardarse el archivo debe de terminar en .py.


\title{\huge\textbf{Variables}}

Una variable es un simbolo al que se asigna un valor. Para crear una variable basta escribir \textbf{"variable"="valor"}. Es importante tener en cuenta que el valor de la variable corresponde al ultimo valor que se le asigno en el programa. Asimismo al crear variables es muy importante indicar sus nombres de manera explicita. 

\title{\huge\textbf{Comandos importantes}}


\textbf{python + tabulador}: indica las versiones de python disponibles en el ordenador.

 
\textbf{python - -version} : indica la version de python que esta en uso.



\end{document}


