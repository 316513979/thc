\documentclass[letterpaper, 12pt, oneside]{article}%para dar formato al documento
\usepackage{amsmath}
\usepackage{graphicx}
\usepackage{xcolor}
\usepackage{enumitem}
\usepackage[utf8]{inputenc}

\title{\Huge Clase 09}
\author{Diego Armando Santillán Arriaga}
\date{17/01/19}


\begin{document}
	\maketitle
	\newpage
\section*{Notas}
Para saber cual es el tipo de una variable se escribe type("variable")
type("variable")
El comando tmp (temporal) permite hacer una asignación temporal a una variable. A continuación se muestra un ejemplo de su uso:
\begin{verbatim}
if a<b:
tmp = b
b = a
a = tmp
r=a%b       
\end{verbatim}
 
\section{Libros en LaTex}
Entre los formatos para escribir en LaTex está el de libro. El formato de libro nos permite incluir portada, introducción, índice y organización por capítulos, entre otras cosas. 

Para agregar un índice, próximo al comando begin se escribe
\begin{verbatim}
\tableofcontents
\end{verbatim}

Para crear un capítulo se utiliza:
\begin{verbatim}
\chapter{"título del capítulo"}
\end{verbatim} 
Automáticamente el capítulo se registra en el índice

Para agregar bibliografía se utiliza:
\begin{verbatim}
\begin{thebibliography}{"número de referencias que se van a registrar}
\end{thebibliography}
\end{verbatim} 
Asimismo para agregar referencias se utiliza bibitem{"libro"} en el caso de libros y textit{"autor, año, etc} en el caso de otros documentos. 
\end{document}