\documentclass[letter paper, 12pt, oneside]{article}
\usepackage{amsmath}
\usepackage{graphicx}
\usepackage{xcolor}
\usepackage{enumitem}
\usepackage[utf8]{inputenc}


\title{\Huge Clase 01}
\author{Diego Armando Santillán Arriaga}
\date{07/01/2019}


\begin{document}
	\maketitle
\newpage
\title{\huge\textbf{ Sistemas Operativos}}

Un sistema operativo es un conjunto de programas que se encargan de controlar el hardware de una computadora. Gracias al sistema operativo nos es posible utilizar aplicaciones y programas en nuestra computadora. 
Existen multiples sistemas operativos. Los mas populares son \textbf{Windows, iOS y Linux}.Este ultimo se puede obtener en alguna de sus distribuciones(conjunto de software ya configurado) que son Fedora, Ubuntu y Debian entre otras. Ademas Linux destaca por ser un sistema operativo de software libre lo cual implica que es posible modificarlo.

\title{\huge\textbf{Linux}}

Como cualquier sistema operativo Linux tiene un entorno grafico, esto es un "menu" en el que se puede navegar para ejecutar programas, abrir y guardar documentos y, en general, administrar la computadora. Sin embargo para poder tener un mayor acceso al sistema operativo es necesario utilizar la \textbf{linea de comandos}. Su funcionamiento consiste en escribir y ejecutar codigos especiales que cumplen funciones determinadas.En el caso de Linux a estos codigos se les llama comandos bash mientras que la linea de comandos es un interprete de comandos bash. Esto se debe a que Linux utiliza bash, que es un programa informatico, para recibir e interpretar ordenes. 

 \title{\huge\textbf{Archivos en Linux}}
 
 
 En Linux el uso de un archivo esta limitado por ciertos permisos. Hay 3 tipos de permisos, de lectura(r), de escritura(w) y de ejecucion(x).
 Estos se pueden otorgar al usuario de la computadora, a un grupo con el que se compartan archivos y al publico en general. Es decir los permisos rwx se asignan a cada uno de estos "sujetos". Es importante tener en cuenta que cada permiso tiene un numero asociado, el de lectura 4, el de escritura 2 y el de ejecucion1. Al final se suman los numeros de todos los permisos que tiene un archivo para un sujeto. Posteriormente los totales obtenidos para cada "sujeto" se colocan juntos de modo que forman un numero de 3 cifras. Esto indicara los permisos que tiene un archivo.  
 
 Para conocer los permisos de un archivo y su numero de 3 cifras correspondiente se utiliza el comando \textbf{ls-l  "nombre del archivo"}. Por otro lado para cambiar los permisos de un documento se utiliza \textbf{chmod nnn "nombre del archivo"}. En el lugar en donde estan las enes se debe de colocar el nuevo numero de 3 cifras, segun los permisos que se quiera asignar a cada "sujeto".
 
 \newpage
 \title{\huge\textbf{Variables de entorno}}
 
 Las variables de entorno definen la forma en que trabaja el sistema y permiten modificar como "se comporta" el shell. Un ejemplo es la variable PATH que permite buscar variables binarias. El comando \textbf{set} se utiliza para observar las variables de entorno. 
 
\textbf{Comandos utiles}



\textbf{uname -a}: arroja informacion sobre el kernel (parte principal del sistema operativo)

\textbf{touch "nombre de archivo"}: crea un archivo vacio.

\end{document}

