\documentclass[letter paper, 12pt, oneside]{article}
\usepackage{amsmath}
\usepackage{graphicx}
\usepackage{xcolor}
\usepackage{enumitem}
\usepackage[utf8]{inputenc}


\title{\Huge Clase 06}
\author{Diego Armando Santillán Arriaga}
\date{14/01/2019}


\begin{document}
	\maketitle
	
\newpage

\textbf{Algoritmo:} proceso de instrucciones finitas.
\\\\
\textbf{Asignacion:}
Una asignación consiste en asignar valores a una variable ya sea a través de una fórmula o directamente. Son distintas de las igualdades. 
Cuando tenemos una asignacion(aparentemente igual que una igualdad matematica) un lado de la igualdad se evalua y el resultado se asigna a la variable del lado contrario.
\\\\
\textbf{If y else:}
Si la condicion del if es verdadera se ejecuta su bloque; de cualquier otra forma se ejecuta el bloque de else. En cualquier caso la instrucción solo se ejecuta una sola vez.
El ciclo if y else se escribe:
\\\\
 \textbf{if "condición":
\\
"Instrucción"
\\
else:
\\"Instruccion"}  \\\\ 
Un bloque son las instrucciones que están indentadas a partir de un cierto nivel (todo lo que esta después de :).
Se pueden colocar tantas funciones if como se quieran. 
Que algo haga "eco" implica que se despliege un resultado (análogo a print). 
\\\\
\textbf{Operadores lógicos:}
Los operadores lógicos se utilizan para concatenar condiciones. Estos son:
\\
\textbf{and:} verdadero si todas las condiciones que une son verdaderas.
\\
\textbf{or:} verdadero si al menos una de las condiciones que une es verdadera. 
\\
\textbf{While:}
While se utiliza para que se ejecute una instrucción o serie de instrucciónes un determinado número de veces. El ciclo while se escribe:
\\\\
\textbf{while "condición":
\\
"instrucción"}
\\\\
Si se desea que  se despliege el resultado de cada ejecución se debe de colocar un comando print dentro del ciclo while. Si solo se quiere desplegar el resultado final el print se escribe fuera del ciclo.  
\end{document}





