\documentclass[letter paper, 12pt, oneside]{article}
\usepackage{amsmath}
\usepackage{graphicx}
\usepackage{xcolor}
\usepackage{enumitem}
\usepackage[utf8]{inputenc}


\title{\Huge Clase 05}
\author{Diego Armando Santillán Arriaga}
\date{10/01/2019}


\begin{document}
	\maketitle
\newpage
\section{Cadenas  en python}


Una cadena en python es un conjunto de caractéres que está delimitado por \textbf{comillas simples}, \textbf{dobles} o \textbf{triples}:


~las cadenas de limitadas por comillas simples (') o dobles (") se despliegan en una sola línea. 


~las cadenas delimitadas por tres comillas simples (''') se despliegan en múltiples lineas. 


Para desplegar una cadena es necesario colocar el comando \textbf{print "cadena de texto"}. Cabe mencionar que los caractéres de la cadena pueden ser de cualquier naturaleza. 


Por otro lado es posible colocar una variable en una cadena y darle formato. Esto se hace reemplazando la variable por el signo \textbf{porcentaje} dentro de la cadena seguido de un código (g, d, 5d, e, E, etc) según el formato que se le quiera dar a la variable. Después al final de la cadena se vuelve a escribir el signo de \textbf{porcentaje} y entre paréntesis las variables de la cadena en el mismo orden en que están escritas. 



\section{Módulos en python}
	

En python un \textbf{módulo} es una biblioteca en donde están definidas una serie de funciones que se deben de importar para utilizar. Para ello se utiliza el comando \textbf{import "nombre del módulo que se va a importar"}. Un ejemplo es la funcion raíz cuadrada: para calcularla es necesario importar la biblioteca math. 

Para utilizar una función de una biblioteca se escribe el \textbf{"nombre de la biblioteca"."nombre de la función"}. Por ejemplo en el caso de la raíz cuadrada se escribe math.sqrt("valor del que se desea calcular la raíz cuadrada").


Por otro lado, es posible crear un módulo. Para ello en un programa (archivo de texto plano) definimos las funciones utilizando los siguientescomandos en el orden indicado:

\begin{enumerate}
\item{def "nombre de la función"(variable independiente 1, variable independiente 2, ...)}

\item{"Escribimos la función"}


\item{return(variable dependiente)} 

\end{enumerate}
Esto se hace por cada función que se quiera construir. Posteriormente se importa el programa al shell utilizando import y el nombre con el que se guardó. Luego para usar las funciones se escribe \textbf{"nombre del documento."nombre de la función"}.

\section{LaTex}


En las primeras líneas del documento se establece el tipo de letra y de documento con documentclass; se instalan los paquetes que se van a utilizar como "amsmath" para instrumentos de matemáticas o "xcolor" para agregar color. 
Para comenzar a redactar el documento se debe de escribir begin{document} y después end{document}. El texto estara escrito entre estos dos comandos. 
Para agregar formato al texto se utilizan comandos como Huge para aumentar el tamano del texto o textbf para escribir en "negritas". 

\end{document}



