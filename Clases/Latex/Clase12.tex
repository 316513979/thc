\documentclass[letter paper, 12pt, oneside]{article}
\usepackage{amsmath}
\usepackage{graphicx}
\usepackage{xcolor}
\usepackage{enumitem}
\usepackage[utf8]{inputenc}


\title{\Huge Clase 12}
\author{Diego Armando Santillán Arriaga}
\date{22/01/2019}


\begin{document}
	\maketitle	
	\newpage
\section{Listas (continuación)}

2 partes en la generación de listas por compresión: 
características de cada elemento de la lista y después como se van a generar dichos elementos.
Formato:
\\\\
"Lista"=["características de cada elemento de la lista" "forma en que se va a obtener cada elemento de l lista (ciclo for)"]
\\\\
La función pprint importada de pprint, le da formato a la impresión de una tabla (las ordena verticalmente). Los comandos involucrados son:
\\\\
\textbf{from pprint import pprint
pprint("nombre de la lista")}
\\\\
Para regresar los elementos de una lista a partir de un índice en una nueva lista (se forma una nueva lista) se utiliza: 
\\\\
\textbf{"nombre de la lista"["índice a partir del cual queremos que nos regrese los elementos":]}
\\\\
Para regresar los elementos de una lista cuyo es índice es anterior a un índice dado en una nueva lista (se crea una nueva lista) se utiliza: 
\\\\
\textbf{"nombre de la lista":["índice del elemento a partir del cual queremos que nos regrese los elementos anteriores"]}
\\\\
Cada vez que hago referencia una sublista esto me genera un nuevo objeto.
Para crear una lista que contenga los elementos de una lista que están entre 2 índices (incluyendo al elemento que tiene el índice que funciona como límite superior).
Dos listas "iguales" pero que tienen diferente nombre (por ejemplo A=[1,2] y B=[1, 2]) son objetos distintos.
Una lista es un objeto. Al asignar una lista a una variable en realidad se asigna la referencia al objeto. Así dos variables que tienen asignadas las mismas listas hacen referencia al mismo objeto. Por esta razón al compararlas el resultado será "True". Sin embargo si introducimos A is B el resultado será falso pues no es la misma referencia. 

"lista".index("valor"): regresa el primer índice en el que está el valor. 

Una coma al final del bloque de un ciclo for omite el salto de línea. Ejemplo:
\\
\begin{verbatim}
for i in range(len(alumnos)):
    for j in range(len(alumnos[i])):
        calificacion = alumnos[i][j]
        print '%4d' %calificacion,

    print
    
>>> 9    8    10    9
    9
    6    9    10    8


\end{verbatim}
\section{Presentaciones en LaTex}
\begin{verbatim}
\documentclass{beamer}
\end{verbatim}
Esta clase de documento sirve para crear presentaciones.
\begin{verbatim}
\begin{frame}
\end{frame}
\end{verbatim}
Permite crear diapositivas (frames).

\begin{verbatim}
\frametitle{}
\end{verbatim}
Título de la diapositiva. 

\begin{verbatim}
\usetheme{"Tema"}
\end{verbatim}
Estas son plantillas de las diapositivas, agregan color y marcos. En temas va el nombre de un tema como por ejemplo: Antibes, AnnArbor, Berkeley, CambridgeUS, Goettingen, Bergen, Dresden, Hannover, Ilmenau, Berlin.  

\begin{verbatim}
\includegraphics[scale="escala"]{"nombre de la imagen y terminación"}
\end{verbatim}
Incluye una imagen. Para que LaTex pueda extraer la imágen de su directorio se necesita:
\begin{verbatim}
\graphicspath{"nombre del directorio en que se encuentra"}
\end{verbatim}

El tema Bergen necesita los siguentes comandos para colocar el autor y la fecha:
\begin{verbatim}
\graphicspath{"nombre del directorio en que se encuentra"}
\end{verbatim}
\def\instertauthorindicator{¿Quién?}
\def\insertdateindicator{Fecha}

\end{document}



