\documentclass[letterpaper, 12pt, oneside]{article}%para dar formato al documento
\usepackage{amsmath}
\usepackage{graphicx}
\usepackage{xcolor}
\usepackage{enumitem}
\usepackage[utf8]{inputenc}

\title{\Huge Problema 9}
\author{Diego Armando Santillán Arriaga}
\date{22/01/19}

\begin{document}
\maketitle
\newpage
\section*{¿Cómo resolví el problema?}
Escogí hacer un programa que determinara entre qué números es divisible un número dado porque pienso que es información relevante lo cual le da utilidad al programa. Para obtener esta idea me basé en ejercicios de Python que encontré en internet. Funciona de la siguiente manera: con un ciclo while una variable, i,  todos los valores entre el 1 y el número del que se están buscando los divisores. A cada valor se le aplica la operación módulo, que efectúa la división entera enre el número y el valor de la variable. El residuo de este cociente se almacena en una variable, r, y con un condicional if se imprime el valor de i si el residuo es igual a 0. 

  
\end{document}