\documentclass[letterpaper, 12pt, oneside]{article}%para dar formato al documento
\usepackage{amsmath}
\usepackage{graphicx}
\usepackage{xcolor}
\usepackage{enumitem}
\usepackage[utf8]{inputenc}

\title{\Huge Problema 6}
\author{Diego Armando Santillán Arriaga}
\date{22/01/19}

\begin{document}
\maketitle
\newpage
\section*{Definir}
El problema consiste en generar una función que calcule el promedio de 10 valores cualesquiera. La fórmula correspondiente es $ m = \frac{v1+v2+v3+v4+v5+v6+v7+v8+v9+v10}{10} $ donde v1, v2, ..., v10 son los valores que se promediarán.
\section*{Analizar y delimitar}
Las entradas y salidas de la función son números reales por lo que este es su dominio y rango.
Para generarla utilizaremos la fórmula mencionada anteriormente con una pequeña modificación: el 10 tendrá punto decimal. Así la expresión que se va a utilizar es: $ m = \frac{v1+v2+v3+v4+v5+v6+v7+v8+v9+v10}{10.0} $.  
\section*{Solución}
El funcionamiento de la solución es muy sencillo: 
se introducen los 10 valores que se van a promediar, estos sustituyen a las variables en la fórmula mencionada anteriormente. El resultado se asigna a la variable p y después su valor se despliega en una cadena de texto que indica que ese es el promedio. 


\end{document}