\documentclass[letter paper, 12pt, oneside]{article}
\usepackage{amsmath}
\usepackage{graphicx}
\usepackage{xcolor}
\usepackage{enumitem}
\usepackage[utf8]{inputenc}


\title{\Huge Lista de preguntas clase 11}
\author{Diego Armando Santillán Arriaga}
\date{22/01/2019}

\begin{document}
	\maketitle
	\newpage
	\textbf{En python ¿qué comando se debe escribir para que una variable sea reconocida como un flotante?}
	\\
	R: float("nombre de la variable")
	\\\\
	\textbf{¿Cuál es la función de la carpeta .gitignore?}
	\\
	Se utiliza para indicar los archivos que no se suben al repositorio.
	\\\\
	Considera la lista L=[1, 5, 7]:
	\\
	\textbf{Escribe un comando que recorra la lista sin utilizar índices:}
	\\
	R: for i in L
	\\\\
	\textbf{¿Qué comando se utiliza para recorrer la lista con sus índices?}
	\\ 
	R: for i in range(len(L)):
	L[i]
	\\\\
	\textbf{En un programa de python que tiene varias instrucciones anidadas ¿Cuál es la primera que se ejecuta?}
	\\
	R: la instrucción que está más anidada. 
	\\\\
	\textbf{Crea una lista que tenga los números del 1 al 5 utilizando la forma condensada}
	\\
	R: a=5; L=[i+1 for i in range(a)]
	\\\\
	\textbf{¿Para qué sirve el comando zip en python?}
	\\
	R: zip crea listas con los elementos de otras listas correspondientes a un mismo índice. 
	\\\\
	\textbf{¿Para que sirven los paréntesis y los corchetes en Python?}
	\\
	R: Los paréntesis se usan en las funciones para agregar argumentos y los corchetes sirven para delimitar una lista o utilizar uno de sus elementos.

\end{document}
	
	
	