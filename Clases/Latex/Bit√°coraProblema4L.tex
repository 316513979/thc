\documentclass[letterpaper, 12pt, oneside]{article}%para dar formato al documento
\usepackage{amsmath}
\usepackage{graphicx}
\usepackage{xcolor}
\usepackage{enumitem}
\usepackage[utf8]{inputenc}

\title{\Huge Problema 4L}
\author{Diego Armando Santillán Arriaga}
\date{22/01/19}

\begin{document}
\maketitle
\newpage
\section*{¿Cómo resolví el problema?}
Escogí crear una lista que contiene los términos de la sucesión de Fibonacci hasta cierto término porque me pareció la aplicación más natural de las listas al problema 4. En ese problema me base para crear este programa. Su funcionamiento es el siguiente. Si se solicitan los 2 primeros términos, utilizando condicionales if y else anidados la función devuelve una lista con 0 ó 0 y 1 según sea el caso. Si se solicitan términos sucesivos, gracias a un condicional else, se activa un ciclo while igual al del problema 4. El ciclo calcula los valores de la sucesión de Fibonacci hasta el término indicado y los va agregando a una lista. Al final la función devuelve la lista.  
\end{document}


