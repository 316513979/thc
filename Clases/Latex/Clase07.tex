\documentclass[letterpaper, 12pt, oneside]{article}%para dar formato al documento
\usepackage{amsmath}
\usepackage{graphicx}
\usepackage{xcolor}
\usepackage{enumitem}
\usepackage[utf8]{inputenc}

\title{\Huge Clase 07}
\author{Diego Armando Santillán Arriaga}
\date{15/01/19}
\\

\begin{document}
	\maketitle
	\newpage
	\section{Notas sobre while}
	While repite una serie de instrucciones un cierto número de veces que depende de la condición que se defina. Si ponemos un comando print indentado al ciclo while se despliega el resultado de cada ejecución; si no está incluido en el ciclo escribe el resultado final. 
	Como funciona:
	"Mientras algo sea verdadero, se realiza una instrucción."
	
	¡Los espacios son importantes en python cuando se definen bloques!
	
	Las funciones se componen de varias partes como (condiciones iniciales, cálculos, arrojar valores, etc).
	
	Para contar cuantas veces se repite while se agrega una variable i=0 y dentro el ciclo al ciclo while se escribe i = i+1
	
	Siempre es más recomendable hacer un programa por partes porque así es más facil corregirlo.
	
	El operador módulo en python es un signo de porcentaje y dado un número x hace esta operacion: (x/2)*2 - x == 0 
	
	\section{LaTex}
	Lo que esta entre signos de dólar se escribe en forma matemática (fórmula, ecuación, etc).
	section hace un encabezado numerado. Si ponemos section* no lo numera. 
\end{document} 




