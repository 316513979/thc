\documentclass[letter paper, 12pt, oneside]{article}
\usepackage{amsmath}
\usepackage{graphicx}
\usepackage{xcolor}
\usepackage{enumitem}
\usepackage[utf8]{inputenc}


\title{\Huge Clase 04}
\author{Diego Armando Santillán Arriaga}
\date{10/01/2019}


\begin{document}
	\maketitle
\newpage
\section{Planteamiento de problemas}


En el siguiente problema aplicaremos el proceso de resolución de problemas visto en la clase pasada: se lanza una pelota al aire y se debe estimar su posición a través del tiempo. 
La función que describe la posición de la pelota es \textbf{$y(t) = v0*t-\frac{1}{2}*t$} donde "y" es la posición en un tiempo de terminado, v0 es la velocidad inicial de la pelota medida en metros sobre segundo y t es el tiempo medido en segundo. De aquí se deduce que la trayectoria de la pelota es una parábola y que el dominio de la función son todos los reales. Sin embargo vamos a delimitar el dominio al intervalo de tiempo \textbf{[0s , 6.93s]} pues no consideraremos las medidas negativas de posición. Por otro lado consideraremos $v0 = 34 m/s$ y $g = 9.81 m/s^{2}$. De aqui procedemos a calcular los valores en los que estamos interesados en python. 


\section{Cálculos en python}

\textbf{Idle}: Idle es un entorno de desarrollo integrado que se utiliza para programar en python. Idle abre \textbf{python shell}. A su vez un  \textbf{shell} es un intérprete de comandos especializado. De esta manera se debe de abrir Idle para poder utilizar python. Para ello se utiliza el comando \textbf{idle}.

\subsection{Observaciones importantes para hacer cálculos en python:}

~Para que una \textbf{división} arroje un resultado decimal es necesario que alguno de los operandos este escrito como decimal. De lo contrario redondea el resultado al menor entero. \textbf{Ejemplo:} $\frac{1}{2} = 0$ pero $\frac{1.0}{2} = 0.5$ ó $\frac{1}{2.0} = 0.5$.
\\\\
~Para guardar un programa es necesario copiarlo del shell a un documento de texto plano utilizando la pestana de "File" del menú de editor. Al guardarse el archivo debe de terminar en .py.


\section{Variables}

Una variable es un símbolo al que se asigna un valor. Para crear una variable basta escribir \textbf{"variable"="valor"}. Es importante tener en cuenta que el valor de la variable corresponde al último valor que se le asignó en el programa. Asimismo al crear variables es muy importante indicar sus nombres de manera explícita. 

\section{Comandos importantes}

\begin{enumerate}

\item{python + tabulador}: Indica las versiones de python disponibles en el ordenador.
 
\item{python - -version} : Indica la versión de python que está en uso.

\end{enumerate}

\end{document}




