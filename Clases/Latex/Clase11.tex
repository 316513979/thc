\documentclass[letter paper, 12pt, oneside]{article}
\usepackage{amsmath}
\usepackage{graphicx}
\usepackage{xcolor}
\usepackage{enumitem}
\usepackage[utf8]{inputenc}


\title{\Huge Clase 11}
\author{Diego Armando Santillán Arriaga}
\date{21/01/2019}


\begin{document}
	\maketitle	
\newpage
\section{Notas:}
Para que una variable pueda ser reconocida como un número flotante debe de escribirse de la siguiente manera:
\\\\
\textbf{float("variable")}

Los archivos .pyc son una versión compilada de un módulo de Python. Son más rápidos. 

\textbf{.gitignore} : Este archivo se crea en gituhub. En él se listan las terminaciónes de los archivos que no nos interesan como por ejemplo .pyc o .aux. Para escribir las terminaciones se utiliza el siguiente formato:

"terminación" (el asterisco sirve para indicar que todos los archivos que tienen esa terminación serán ignorados)

Para importar varias funciones de un módulo en un solo comando utilizamos

from "nombre del módulo" import * 


\section{Listas (continuación):}
Se pueden escribir variables dentro de un a lista. Al dar valores a estas variabes, automáticamente se asignan a su lugar correspondiente en la lista. 

Para crear una lista que tiene n elementos contados desde el cero se  utiliza:

for i in range("número"):

Esta lista va desde el 0 hasta el número n-1 


for i in  range(len("lista"))

Este ciclo toma cada elemento de una lista que tiene los índices de la lista "lista"

La expresión que está más anidada es la primera que se ejecuta

\begin{verbatim}
>>>L1
[0, 10, 15, 20]
>>> for i in range(len(L1)):
L1[i] += 5

>>>L1
[5, 15, 20, 25]
\end{verbatim}

"nombre de la lista"[i (índice de algun elemento de la lista)] regresa el iésimo elemento de la lista. 

Para utilizar en un ciclo for un índice y su valor correspondiente en la liste se utiliza enumerate de la siguiente forma:

for "índice", "variable que representa el valor correspondiente" in enumerate("lista"):

Una forma condensada de crear listas es:

"nombre de la lista"= ["fórmula o variable" + ciclo for]

Ejemplos
\begin{verbatim}
>>> n=12; gradosC = [-5 + i*0.5 for i in range(n) ]
>>> gradosC
[-5.0, -4.5, -4.0, -3.5, -3.0, -2.5, -2.0, -1.5, -1.0, -0.5, 0.0, 0.5]
>>> C_mas_7 = [ C+7 for C in gradosC]
>>> C_mas_7
[2.0, 2.5, 3.0, 3.5, 4.0, 4.5, 5.0, 5.5, 6.0, 6.5, 7.0, 7.5]
\end{verbatim}

zip("lista1","lista2) toma los valores de ambas listas correspondientes al mismo índice (también aplica para más de dos listas, de hecho se forman tuplas).

¿Para que sirven los paréntesis, llaves y corchetes en Python?
paréntesis = en una función permite agregar los argumentos.
corchetes = sirven por acceder a un elemento de una lista o definirla.
s
Con las listas se pueden formar tup0las. 
\end{document}


