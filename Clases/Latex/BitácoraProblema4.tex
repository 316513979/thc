\documentclass[letterpaper, 12pt, oneside]{article}%para dar formato al documento
\usepackage{amsmath}
\usepackage{graphicx}
\usepackage{xcolor}
\usepackage{enumitem}
\usepackage[utf8]{inputenc}

\title{\Huge Problema 4}
\author{Diego Armando Santillán Arriaga}
\date{22/01/19}

\begin{document}
\maketitle
\newpage
\section*{Definir}
Este problema consiste generar una función que devuelva el enésimo término de la sucesión de Fibonacci. Los primeros 2 términos de la sucesión son 0 y 1 respectivamente. El resto de los términos se construyen sumando los dos anteriores. 
\section*{Analizar y delimitar}
La solución estará diseñada para los enteros positivos y el 0, este será el dominio y rango de la función. Los instrumentos útiles para resolver el problema son variables, el ciclo while y posiblemente los condicionales if y else. También necesita haber una variable que actué como contador para limitar la repetición de la suma de los términos.  
\section*{Solución}
La variable independiente de la función es n, el número de término. Los dos primeros términos se generan con condicionales if y else anidados en otro condicional if y else: como estos corresponden a los términos 1 y 2 el primer if se activa si n<3. Luego el condicional anidado devuelve 0 si n=1 ó 1 si n=2. A partir de aquí los demás términos se generan con una suma.   Este bloque se activa si 3<=n, es decir, si no se cumplió el primer if. Entonces se reasignan valores a las variables a=0 y b=1 de tal manera que los términos de la sucesión se asignan a la variable b cuyo valor devuelve la función. 

\section*{Comentarios}
Para poder construir el programa tuve que basarme en un ejercicio similar que encontre en el link https://www.w3resource.com/python-exercises/python-conditional-exercise-9.php porque no se me ocurrían las operaciones a efectuar entre las variables para encontrar los términos. Gracias al ejercicio de internet me di cuenta de que como había que organizar las variables para que se generaran los términos
de la sucesión.
	
\end{document}