\documentclass[letter paper, 12pt, oneside]{article}
\usepackage{amsmath}
\usepackage{graphicx}
\usepackage{xcolor}
\usepackage{enumitem}
\usepackage[utf8]{inputenc}


\title{\Huge Lista de preguntas semana 2}
\author{Diego Armando Santillán Arriaga}
\date{14/01/2019}

\begin{document}
	\maketitle
	\newpage
	\textbf{¿Qué es un algoritmo?}
	\\
	R: Es un proceso de instrucciones finitas.
	\\\\
	\textbf{Escriba un ejemplo en el que utilize if y else:}
	\\
	R: if "condición":
	\\
	"Instrucción"
	\\
	else:
	\\"Instruccion"
\\\\	
	\textbf{Mencione los operadores logicos y cuando son verdaderos:}
\\
and: verdadero si todas las condiciones que une son verdaderas.
\\
or: verdadero si al menos una de las condiciones que une es verdadera. 
\\\\
\textbf{¿Cuántas veces se ejecutan las instrucciones de un ciclo while?}
\\
R: Tantas veces como se especifique en la condicion que se coloca al lado de while. 
\\\\
\textbf{¿Cuándo y por qué son importantes los espacios en python?}
\\
R: En python los espacios son importantes cuando formamos bloques y sirven para indentar instrucciones a un bloque.
\textbf{¿Qué es lo que devuelve la operación módulo?}
\\
R: La operación módulo devuelve el residuo que se genera con la división entera de 2 números. 
\\\\
\textbf{¿Qué sucede si colocamos un comando print dentro delbloque de un ciclo while?}
\\
R: Se imprime el resultado de cada repetición del ciclo while. 
\\\\
\textbf{En LaTex ¿qué símbolos deben delimitar una expresión para que se muestre como expresión matemática?}  
\\
R: Se escribe el signo \begin{verbatim}
$
\end{verbatim}
\textbf{En la terminal, mientras un proceso está realizándose ¿qué sucede si se presiona ctrl y c?}
R: El proceso se detiene definitivamente.
\\\\
\textbf{¿Para qué sirve ; en python?}
\\
R: Se utiliza para separar instrucciones cuando se escriben en una misma línea. 
\\\\
\textbf{¿Por qué es más conveniente importar de una biblioteca solo las funciones que vamos a utilizar?}
\\
R: Porque entre más información importemosa un programa, más memoria RAM ocupa y por lo tanto su ejecución se alenta. 
\\\\
\textbf{¿Para qué sirve el comando input en python?}
\\
R: Permite que el usuario seleccione el valor de una variable. 
\\\\
\textbf{¿Qué es una clase?}
\\
R: Una clase en un conjunto de objetos que comparten características generales.
\\\\
\textbf{¿Cuál es la diferencia entre un método y una función?}
\\
R: El método depende de un objeto y siempre van ligados, mientras que la función no depende de un objeto. 
\\\\
\textbf{Menciona 3 tipos de variables diferentes:}
\\
R: int, float, string, list.
\\\\
\textbf{Al crear un libro en LaTex ¿Qué comando se utiliza para incluir un índice?}
\\
R: \begin{verbatim}
\tableofcontents
\end{verbatim}
\textbf{En python ¿Cuales son los posibles resultados de comparar una variable con otra?}
R: True o False. 
\\\\
\textbf{En una lista de python ¿desde qué número se comienzan a indexar los elementos?}
\\
R: Desde el 0. 
\\\\
\textbf{En python ¿Cuál es la limitante de crear listas utilizando range?}
\\
R: Con range no se pueden hacer listas cuyos elementos tengan una diferencia flotante. 
\\\\
\textbf{En la siguiente instrucción: "for i in L" ¿qué representa la i?}
\\
R: La i representa a cada elemento de la lista L.

\end{document}


