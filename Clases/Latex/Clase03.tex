\documentclass[letter paper, 12pt, oneside]{article}
\usepackage{amsmath}
\usepackage{graphicx}
\usepackage{xcolor}
\usepackage{enumitem}
\usepackage[utf8]{inputenc}

\title{\Huge Clase 03}
\author{Diego Armando Santillán Arriaga}
\date{09/01/2019}


\begin{document}
	\maketitle
\newpage
\section{Como subir archivos a Github}

Para subir un archivo a github debe de estar guardado en el repositorio de nuestra computadora. Una vez hecho esto desde el repositorio ejecutamos los siguientes comandos en el orden indicado:

\begin{enumerate}
\item{git init}: Esto conecta el repositorio de nuestro ordenador con github.

\item{git add *}: Este comando se usa para agregar archivos nuevos y modificaciones. El asterisco indica que se agregarán todos los archivos y modificaciones que se hayan hecho. También es posible agregar un solo archivo utilizando \textbf{git add "nombre del archivo"}.
 

\item{git commit}: Con este comando se ordena a git que registre modificaciones y archivos nuevos por lo cual pide agregar un comentario en el que indiquemos que cambios y archivos se agregaron al repositorio. Podemos escribir directamente el comentario utilizando \textbf{git commit -m "comentario"}. 

\item{git push}: la ejecución de este comando sube todos los archivos nuevos y modificaciones pero para ello, cuando se nos indique debemos escribir nuestro nombre de usuario y contraseña de Github.

\end{enumerate}

\section{Comandos importantes}

\begin{enumerate}
\item{mkdir}: Crea un directorio.


\item{mkdir -p directorio1/directorio2/...}: Crea directorios anidados.


\item{cat "nombre del archivo"}: Muestra el contenido de un archivo.


\item{ls}: Muestra directorios.


\item{ls -la}: Muestra informacion sobre directorios y archivos (en el extremo izquierdo "d" significa directorio y "-" archivo).


\item{file "nombre de archivo"}: Indica el tipo de un archivo. 

\item{file *}: Indica el tipo de archivo de todos los archivos de un directorio. 
\end{enumerate}

\section{Proceso para resolver un problema}

\begin{enumerate}
\item{Definir o entender el problema}


\item{Analizarlo y delimitarlo}: analizar cómo se puede resolver y que instrumentos se tienen disponibles para ello.


\item{Revisar soluciones ya existentes}


\item{Describir la solución detalladamente}


\item{Presentar una solución general}
\end{enumerate}



\end{document}