\documentclass[letterpaper, 12pt, oneside]{article}%para dar formato al documento
\usepackage{amsmath}
\usepackage{graphicx}
\usepackage{xcolor}
\usepackage{enumitem}
\usepackage[utf8]{inputenc}

\title{\Huge Problema 1L}
\author{Diego Armando Santillán Arriaga}
\date{22/01/19}

\begin{document}
\maketitle
\newpage
\section*{¿Cómo resolví el problema?}
Escogí desplegar los residuos generados al calcular el máximo común divisor con el algoritmo de Euclides porque considero que es información interesante. Por ejemplo a partir de ella se pueden identificar patrones o sucesiones.
El programa utiliza condicionales if y else para determinar el orden en el que deben de ir las variables pues la división no es conmutativa. Luego en el bloque del condicional se calcula cada residuo mayor a 0 con un ciclo while y se van agregando a una lista. La función devuelve esta lista. 
  
\end{document}