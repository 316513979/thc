\documentclass[letterpaper, 12pt, oneside]{article}%para dar formato al documento
\usepackage{amsmath}
\usepackage{graphicx}
\usepackage{xcolor}
\usepackage{enumitem}
\usepackage[utf8]{inputenc}

\title{\Huge Problema 3L}
\author{Diego Armando Santillán Arriaga}
\date{22/01/19}

\begin{document}
\maketitle
\newpage
\section*{¿Cómo resolví el problema?}
Escogí desplegar en una lista varias temperaturas y su equivalencia porque es un buen ejercicio para practicar el tratamiento de varios datos al mismo tiempo. Para hacer el programa me inspiré en los ejercicios de la clase. 
El programa esta formado por 2 funciones que son análogas. Solamente cambian las fórmulas que se utilizan. En la función primero se calcula la diferencia que habrá entre cada temperatura de tal manera que se tengan 10 temperaturas medidas en una escala. Después hace una lista que contiene estas temperaturas. Luego con un ciclo for se calcula la equivalencia de cada temperatura en la otra escala y se imprimen ambas medidas. En el script con condicionales if y else dependiendo de las entradas del usuario se activa la función que convierte de grados centígrados a fahrenheit o la contraria. 
\end{document}



