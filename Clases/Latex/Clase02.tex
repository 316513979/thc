\documentclass[letter paper, 12pt, oneside]{article}
\usepackage{amsmath}
\usepackage{graphicx}
\usepackage{xcolor}
\usepackage{enumitem}
\usepackage[utf8]{inputenc}
\graphicspath{Imágenes/}

\title{\Huge Clase 02}
\author{Diego Armando Santillán Arriaga}
\date{08/01/2019}


\begin{document}
	\maketitle
\newpage
\title{\huge\textbf{Comandos importantes}}

\textbf{top} Arroja informacion sobre la computadora como CPU's y su actividad.

\textbf{less "archivo"} Muestra el contenido de un archivo paginado.


\title{\huge\textbf{Directorio}}


El directorio es el sistema de archivos del sistema operativo, en el podemos econtrar otros archivos o directorios.Para cambiar de directorio se utiliza el comando \textbf{cd "directorio"}. Algunos directorios importantes son

\textbf{/} Directorio raiz


\textbf{lib64} Bibliotecas de 64 bits


\textbf{bibliotecas}


\textbf{home} Usuarios


\textbf{media} Dispositivos conectados


\textbf{mnt} Webcams, discos duros externos


\textbf{..} Directorio anterior


\textbf{/dev}Aparatos conectados con el equipo

\title{\huge\textbf{Como utilizar un repositorio en github}}


Github es un sistema de control de versiones. Nos permite crear archivos y administrar las modificaciones que les vayamos haciendo. Tambien funciona como plataforma para trabajar en conjunto con otros usuarios.

Para poder subir nuestros archivos a Github es necesario crear un repositorio. Esto es, a grandes rasgos, una carpeta en donde se guardan nuestros archivos. Una vez que se tiene una cuenta en Github el proceso para crear un repositorio es el siguiente. En la pagina de Github se genera un repositorio lo cual arroja un URL. Luego en el equipo se abre una carpeta en la que se guardara el repositorio y, desde la carpeta creada se ejecuta el comando \textbf{git init}. Despues se ejecuta \textbf{git clone "URL del repositorio"}, \textbf{git config - -global user.email "email que se registro en github"} y \textbf{git config - -global user.name "nombre de usuario de github"} y listo, la carpeta de nuestra computadora esta vinculada a nuestro repositorio de github. Cabe mencionar que solo se realiza este proceso una vez. En adelante la carpeta de nuestro ordenador es un repositorio.  
\end{document}

