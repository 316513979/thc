\documentclass[letter paper, 12pt, oneside]{article}
\usepackage{amsmath}
\usepackage{graphicx}
\usepackage{xcolor}
\usepackage{enumitem}
\usepackage[utf8]{inputenc}


\title{\Huge Clase 04}
\author{Diego Armando Santillán Arriaga}
\date{10/01/2019}


\begin{document}
	\maketitle
\newpage
\title{\huge\textbf{Cadenas  en python}}


Una cadena en python es un conjunto de caracteres que esta delimitado por \textbf{comillas simples}, \textbf{dobles} o \textbf{triples}:


~las cadenas de limitadas por comillas simples (') o dobles (") se despliegan en una sola linea. 


~las cadenas delimitadas por tres comillas simples (''') se despliegan en multiples lineas. 


Para desplegar una cadena es necesario colocar el comando \textbf{print "cadena de texto"}. Cabe mencionar que los caracteres de la cadena pueden ser de cualquier naturaleza. 


Por otro lado es posible colocar una variable en una cadena y darle formato. Esto se hace escribiendo en la cadena en el lugar de la variable el signo de \textbf{porcentaje} seguido de un codigo (g, d, 5d, e, E, etc) segun el formato que se le quiera dar a la variable. Despues al final de la cadena se vuelve a escribir el signo de \textbf{porcentaje} y entre parentesis las variables de la cadena el mismo orden en que estan escritas. 



\title{\huge\textbf{Modulos en python}}
	

En python un \textbf{modulo} es una biblioteca en donde estan definidas una serie de funciones que se deben de importar para utilizar. Para ello se utiliza el comando \textbf{import "nombre del modulo que se va a importar"}. Un ejemplo es la funcion raiz cuadrada: para calcularla es necesario importar la biblioteca math. 

Para utilizar una funcion de una biblioteca se escribe el \textbf{"nombre de la biblioteca"."nombre de la funcion"}. Por ejemplo en el caso de la raiz cuadrada se escribe math.sqrt("valor del que se desea calcular la raiz cuadrada").


Por otro lado, es posible crear un modulo. Para ello en un programa (archivo de texto plano) definimos las funciones utilizando los siguientescomandos en el orden indicado:


\textbf{def "nombre de la funcion"(variable independiente 1, variable independiente 2, ...)}

\textbf{"Escribimos la funcion"}


\textbf{return(variable dependiente)} 


Esto se hace por cada funcion que se quiera construir. Posteriormente se importa el programa al shell utilizando import y el nombre con el que se guardo. Luego para usar las funciones se escribe \textbf{"nombre del documento."nombre de la funcion"}.


\newpage
\title{\huge\textbf{LaTex}}


En las primeras lineas del documento se establece el tipo de letra y de documento con documentclass; se instalan los paquetes que se van a utilizar como "amsmath" para instrumentos de matematicas o "xcolor" para agregar color. 


Para comenzar a redactar el documento se debe de escribir begin{document} y despues end{document}. El texto estara escruto entre estos dos comandos. 
Para agregar formato al texto se utilizan comandos como Huge para aumentar el tamano del texto o textbf para escribir en "negritas". 

\end{document}

