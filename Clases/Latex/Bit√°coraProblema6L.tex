\documentclass[letterpaper, 12pt, oneside]{article}%para dar formato al documento
\usepackage{amsmath}
\usepackage{graphicx}
\usepackage{xcolor}
\usepackage{enumitem}
\usepackage[utf8]{inputenc}

\title{\Huge Problema 6L}
\author{Diego Armando Santillán Arriaga}
\date{22/01/19}

\begin{document}
\maketitle
\newpage
\section*{¿Cómo resolví el problema?}
Como el problema 6 la función de este problema también calcula el promedio de 10 números. La diferencia es que se utilizan listas para hacer el cálculo.
Los 10 números dados por el usuario se guardan en una lista. Luego con un ciclo for cada número se multiplica por 0.1, lo cual equivale a dividirlo entre 10.0, y el resultado se suma a una variable. Este proceso es equivalente a sumar todos los números y después dividirlos entre 10. La función devuelve el valor final de la variable. Cabe mencionar que esta forma de calcular un promedio también se puede utilizar para calcular un promedio ponderado. 
\end{document}